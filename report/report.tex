\documentclass[11pt]{article}

\usepackage{latexsym}
\usepackage{amsmath}
\usepackage{amssymb}
\usepackage{amsthm}
\usepackage{graphicx}
\usepackage{wrapfig}
\usepackage{pseudocode}
\usepackage{url}
\usepackage[backref, colorlinks=true, citecolor=red, urlcolor=blue, pdfauthor={Jyh-Ming Lien}]{hyperref}


\newcommand{\handout}[5]{
  \noindent
  \begin{center}
  \framebox{
    \vbox{
      \hbox to 5.78in { {\bf } \hfill #2 }
      \vspace{4mm}
      \hbox to 5.78in { {\Large \hfill #5  \hfill} }
      \vspace{2mm}
      \hbox to 5.78in { {\em #3 \hfill #4} }
    }
  }
  \end{center}
  \vspace*{4mm}
}

\newcommand{\lecture}[4]{\handout{#1}{#2}{#3}{}{Report for #1}}

\newtheorem{theorem}{Theorem}
\newtheorem{corollary}[theorem]{Corollary}
\newtheorem{lemma}[theorem]{Lemma}
\newtheorem{observation}[theorem]{Observation}
\newtheorem{proposition}[theorem]{Proposition}
\newtheorem{definition}[theorem]{Definition}
\newtheorem{claim}[theorem]{Claim}
\newtheorem{fact}[theorem]{Fact}
\newtheorem{assumption}[theorem]{Assumption}

% 1-inch margins, from fullpage.sty by H.Partl, Version 2, Dec. 15, 1988.
\topmargin 0pt
\advance \topmargin by -\headheight
\advance \topmargin by -\headsep
\textheight 8.9in
\oddsidemargin 0pt
\evensidemargin \oddsidemargin
\marginparwidth 0.5in
\textwidth 6.5in

\parindent 0in
\parskip 1.5ex
%\renewcommand{\baselinestretch}{1.25}

\begin{document}

\lecture{Advance Algorithm Programming Assignment 1 }{Fall 2015}{Your name}{Moran Kim}


\section{Implementation Details}
To implement Delaunay triangulation in 3-dimension, I used 'QHull()' to compute covex hull of the vertices in 4-dimension.\\
Given set of vertices $V=\{(v_x,v_y,v_z)\in\mathbb{R}^3|\hspace{0.15cm} for\hspace{0.15cm} v_x,v_y,v_z\in\mathbb{R}\}$, we lift up the points v$\in V$ by adding additional component $V_m=(v_x)^2+(v_y)^2+(v_z)^2$. Let us denote $V4={(v_x,v_y,v_z,v_m)\in\mathbb{R}^4}$. Then we can construct a convex hull of V4 which is a composition of tetrahedrons. After computing the convex hull of V4, we select the vertices of the tetrahedron whose facing direction is downside. After storing the information of that selected vertices $v_{selected}\in V4$, we draw tetrahedron of $v_{selected}$ with the part of Qhull() which is Delaunay().


\section{Example Output}

\section{Known bugs/limitations}

\bibliographystyle{plain}
\bibliography{report}

\end{document}


